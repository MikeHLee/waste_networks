\documentclass[12pt]{article}
\usepackage[utf8]{inputenc}
\usepackage{amsmath}
\usepackage{graphicx}
\usepackage{hyperref}
\usepackage{algorithm}
\usepackage{algorithmic}
\usepackage{natbib}

\title{Causal Analysis of Waste Networks in Supply Chains: A Bayesian Approach}
\author{Michael Lee}
\date{\today}

\begin{document}

\maketitle

\begin{abstract}
This paper presents a novel approach to analyzing waste in supply chain networks using a combination of graph theory and Bayesian causal analysis. We introduce a framework that models supply chains as directed graphs with waste attributes and employs Bayesian networks to discover and quantify causal relationships between various factors affecting waste generation. Our results demonstrate that temporal factors, particularly storage time, have the strongest causal effect on waste accumulation, followed by environmental factors such as temperature and humidity. This work provides valuable insights for optimizing supply chain efficiency and reducing waste through targeted interventions.
\end{abstract}

\section{Introduction}
Supply chain waste represents a significant challenge in modern logistics and distribution systems. While traditional approaches focus on descriptive analytics of waste patterns, understanding the causal relationships between various factors and waste generation remains a crucial yet understudied area. This paper introduces a comprehensive framework that combines network analysis with Bayesian causal inference to identify and quantify the drivers of waste in supply chain networks.

\section{Methodology}
\subsection{Network Model}
We represent the supply chain as a directed graph $G = (V, E)$, where vertices $V$ represent entities (e.g., growers, distributors, stores) and edges $E$ represent relationships between entities. Each vertex $v \in V$ is characterized by attributes including capacity and waste rate. Similarly, each edge $e \in E$ carries attributes such as flow capacity and transportation waste metrics.

\subsection{Causal Analysis Framework}
Our causal analysis employs a Bayesian network approach using PyMC for probabilistic modeling. The framework:
\begin{itemize}
    \item Constructs a directed acyclic graph (DAG) representing causal relationships
    \item Uses MCMC sampling to estimate posterior distributions
    \item Calculates effect sizes and uncertainties through bootstrap analysis
\end{itemize}

\section{Results}
\subsection{Network Analysis}
Analysis of our test network revealed that the minimum waste path from grower to store accumulates 19.5\% total waste, with the following breakdown:
\begin{itemize}
    \item Node-specific waste: Store (8\%), Grower (5\%), Distributor (3\%)
    \item Transportation waste: 2\% and 1.5\% for respective segments
\end{itemize}

\subsection{Causal Effects}
Our Bayesian analysis identified the following causal effects on waste amount:
\begin{itemize}
    \item Storage time: 0.320 (±0.017)
    \item Temperature: 0.240 (±0.025)
    \item Humidity: 0.160 (±0.022)
\end{itemize}

\section{Discussion}
The results suggest that temporal factors, particularly storage time, have the strongest causal influence on waste generation. This finding has important implications for supply chain optimization, suggesting that reducing storage time could be more effective than environmental control measures in minimizing waste.

\section{Conclusion}
This work demonstrates the value of combining network analysis with Bayesian causal inference in understanding waste generation in supply chains. Future work will focus on extending the framework to handle dynamic networks and incorporating additional environmental and operational variables.

\bibliographystyle{plain}
\bibliography{references}

\end{document}
